\documentclass{article}
\usepackage[utf8]{inputenc}
\usepackage[T2A]{fontenc}
\usepackage[russian]{babel}
\usepackage[T1]{fontenc}


\title{BrickGame v2.0}

\date{May 2024}

\begin{document}

\maketitle
\section{Змейка}
\subsection{Описание}
Программа разработана на языке программирования С++ в парадигме объектно-ориентированного программирования. Консольный интерфейс разработан с помощью библиотеки \textbf{ncurses}. 
Десктопный интерфейс разработан на базе одной GUI-библиотеки Qt.
Игровое поле — десять «пикселей» в ширину и двадцать «пикселей» в высоту.

\subsection{Управление}
Для управления поддерживаются следующие кнопоки на физической консоли:
\begin{itemize}
    \item 'ENTER' — Начало игры,
    \item 'P' — Пауза,
    \item 'ESC' — Завершение игры,
    \item Стрелка влево — поменять направление движения,
    \item Стрелка вправо — поменять направление движения,
    \item Стрелка вниз — поменять направление движения,
    \item Стрелка вверх — поменять направление движения,
    \item 'Z' — Действие (ускорение движения змейки).

\end{itemize}

\subsection{Конечный автомат}
КА змейки состоит из следующих состояний:
\begin{itemize}
    \item StartGame — состояние, в котором игра ждет, пока игрок нажмет кнопку готовности к игре (Enter).
    \item Spawn — состояние, в котором спавнится яблоко.
    \item Moving — основное игровое состояние с обработкой ввода от пользователя — поменять направление змейки.
    \item Shifting — состояние, в которое переходит игра после истечения таймера. В нем змейка делает движение.
    \item End — Если игрок нажимает на кнопку 'Esc', то игра завершается.
\end{itemize}

\subsection{Build}
\begin{itemize}
    \item make install — Сборка BrickGame_v2.0
    \item make uninstall — Удаление программы
    \item make dist — Архивирование проекта
    \item make gcov_report — Формирование отчёта в виде html страницы
    \item make leaks — Проверка на утечки памяти
    \item make style — Форматирование
\end{itemize}

\end{document}
